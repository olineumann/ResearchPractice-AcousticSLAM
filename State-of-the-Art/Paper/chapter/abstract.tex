\begin{abstract}
    Simultaneous localization and mapping (SLAM) became one of the most important
    research fields in the last two decades. There are also recent approaches
    facing SLAM with the use of physical phenomena. That is a promising step
    for many use cases, like indoor navigation. One challenging part of those
    approaches is to switch from discrete landmarks to continuous functions.
    This function often needs to be approximated for SLAM and also for monitoring
    usages. In this paper the basic knowledge of SLAM is explained and the mathematics
    behind gaussian processes is introduced. Moreover, recent approaches facing
    SLAM in combination of monitoring, using physical phenomena for SLAM in real time
    and modeling uncertainty in the input of gaussian processes, will be presented.
    At the end, open questions which could be faced in further proceeding are discussed.
\end{abstract}

%\begin{IEEEkeywords}
%    SLAM, monitoring, physical phenomenon, gaussian process
%\end{IEEEkeywords}