\section{Conclusion}
When facing the problem of SLAM using physical phenomena one challenge is
to handle continuous functions in contrast to discrete landmarks. Moreover,
the physical phenomena need to be approximated to use in case of SLAM. 
For that, gaussian processes are needed.

In chapter \ref{chap:slam} a current state-of-the-art graph-based SLAM
techniques is discussed. This approach is proven but it's only working
in case of landmarks. It should be possible to formulate the problem
of SLAM using physical phenomena in a graph-based manner. This could
be faced in later proceedings.

A terrain field based SLAM approach was discussed in \ref{chap:terrain_field}.
It showed that a variety of phyiscal phenomena could be used for a SLAM approach.
In further studies different phenomena could be observed and analyzed how suitable
they are for SLAM approaches. E.g. texture of a subsurface measured by a camera,
emission in a city or environmental sound.

Thinking of different suitable phyiscal phenomena would also lead to the question
how to handle phenomena which change in time. Imagine autonomous submarine using
sea streams for SLAM to monitor a gas pipeline. But sea streams won't be stable
in time. How to still use sea streams for SLAM also would need more research to do.

Another interesting thought is using only gaussian processes with uncertain input
for SLAM without any common SLAM techniques like Kalman filter or graph-based 
approaches. The idea would be that all measurements are taken at an uncertain
position which is the input for the gaussian process. To use gaussian processes
for that case, the uncertainty in the input has to decrease. Current researchers
often only face the problem of getting more accurate in the output but not in the
input data. Equally like the presented paper from Damianou et al. in chapter 
\ref{chap:uncertain_inputs}.

As mentioned before, different phenomena could be used for SLAM. Also, it could 
be an aim to monitor that phenomena. But when it comes to combining monitoring
and SLAM there has to be more work to do. In \ref{chap:complete_coverage} a
traditional SLAM approach with focus on monitoring was introduced, but how to
deal with phenomena changing over time (so they need to be monitored)? The certainty 
in the measurements should decrease over time, because the measurements getting older. 
So, there are also questions to answer.

In conclusion, there are many different problems to solve. Thinking of combining
SLAM with physical phenomena and the aim of monitoring these. Also, getting away
from traditional SLAM approaches and only using gaussian processes with uncertain
input or just consider new suitable physical phenomena for SLAM as mentioned before.