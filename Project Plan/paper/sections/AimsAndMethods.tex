\section{Project Aims}

The overall project goal is to implement a SLAM approach based on the RIR with a mobile platform equipped with an IMU, speaker, and microphone. To reach that, intermediate targets have to be defined. At first, the aim is to show that the RIR is a continuous function also for moving speakers not only for stationary ones which is already shown by Dokmanic \cite{dokmanic_roomrecslam_2016} as described in Section \ref{sec:acoustic_slam}. For that, data of a moving agent has to be recorded which also can be a human holding the equipment and moving around for first tests. 

Section \ref{sec:acoustic_slam} is showing that the RIR is a continuous function in case of a static speaker. This project wants to show that this is also the case for a speaker and microphone mounted on the same object. After that, based on the RIR a continuous function suitable for real-time SLAM has to be found. Therefore several approaches should be compared. For example, using only the most significant peaks. The RIR itself couldn't be used because of performance issues. When a function suitable for SLAM is found, this function needs to be approximated with a Gaussian process (GP) for SLAM usage. Suitable kernel functions for GP should be compared and the best fitting parameter set up for the kernel should be found. In the end, the SLAM approach needs to be implemented and compared to other SLAM approaches which are based on continuous functions.

When a function suitable for real-time SLAM is found, the SLAM approach has to be implemented, tested, and validated. For implementation, other approaches for continuous SLAM described in Chapter \ref{chap:stat_of_the_art} can be used for orientation. For testing purposes, the performance of the implemented SLAM approach should be examined. Where the performance could be the error or real-time requirements. When it comes to evaluation, the approach could be compared to other approaches that are described in Chapter \ref{chap:stat_of_the_art}. Especially to the one from Dokmanic \cite{dokmanic_roomrecslam_2016} where the project is largely based on.