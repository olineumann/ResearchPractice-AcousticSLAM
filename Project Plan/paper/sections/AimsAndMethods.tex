\section{Project Aims}

The overall project goal is to implement a SLAM approach based on the RIR with a mobile platform equipped with an IMU, speaker, and microphone. To reach that, intermediate targets have to be defined. At first, the aim is to show that the RIR is a continuous function also for moving speakers not only for stationary ones which is already shown by Dokmanic \cite{dokmanic_roomrecslam_2016} as described in section \ref{sec:acoustic_slam}. For that, data of a moving agent has to be recorded which also can be a human holding the equipment and moving around for first tests. 

After the knowledge that the RIR is continuous for the moving case, a function suitable for approximation with Gaussian Processes (GP) and SLAM has to be derived. The reason why not using the RIR itself is that, depending on the window size, GP would have a bad performance. As shown in section \ref{sec:acoustic_slam} the peaks are most interesting. That leads to the attempt using the position of the top ten peaks which would increase the performance a lot. But also other approaches have to be examined. For example, using the top lowest magnitudes when thinking of standing and extinguishing waves.

When a function suitable for real-time SLAM is found the SLAM approach has to be implemented, tested, and validated. For implementation, other approaches for continuous SLAM described in chapter \ref{chap:stat_of_the_art} can be used for orientation. For testing purposes, the performance of the implemented SLAM approach should be examined. Where the performance could be the error or real-time requirements. When it comes to valuation, the approach could be compared to other approaches that are described in chapter \ref{chap:stat_of_the_art}. Especially to the one from Dokmanic \cite{dokmanic_roomrecslam_2016} where the project is largely based on.