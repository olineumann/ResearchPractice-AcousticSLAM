\section{Pre-Studies}
\label{chap:pre_studies}

Based on the work of Dokmanic et al. \cite{dokmanic_roomrecslam_2016} the question is now if the RIR will also be suitable for SLAM when the speaker is also mounted on the robot. In the case of Dokmanic et al. \cite{dokmanic_roomrecslam_2016} described in section \ref{sec:acoustic_slam}, the RIR is a continuous function that won't change until the speaker moves or the room will change. In this project, the speaker will move along with the robot. So the RIR as a 2D function over the room will change every time. In the pre-studies, it will be investigated if the RIR is suitable for SLAM when measuring the RIR at the position of a moving speaker.

But first a short recap of the room impulse response (RIR). The principle of sending a signal in a room and recording it with a microphone could be treated as an electrical system where the sent signal is the input and the recorded signal the output of the electrical system. In the concept of electrical systems, the output $y(t)$ depends on the input $x(t)$ and a function $g(t)$ which is convolved with the input in time space. So it follows
$$
y(t) = (g * x)(t) \text{.}
$$

When now stimulating the system with an infinite small function also called impulse, the function $g(t)$ is called the impulse response. Normally the impulse response isn't calculated in time space because of the convolution operator. To calculate the impulse response the input and output is transformed in frequency space with Fourier transformation. In frequency space, the convolution becomes a multiplication and the impulse response can be calculated by dividing the output $y(\omega)$ by the input $x(\omega)$. Which is the same like
$$
y(\omega) = g(\omega) x(\omega) \Rightarrow g(\omega) = \frac{y(\omega)}{x(\omega)}\text{.}
$$

Obviously, the input signal can't be infinitely small. Also, because of the time-frequency uncertainty relation, an impulse has to have a certain length if a wide frequency band wants to be stimulated. For the pre-studies, we decided to use a chirp signal which is a sinusoidal function with a rise in frequency. Before calculating the impulse response, we decided to window the output with a Tukey window. The windows are overlapping slightly so that the area under the overlap is one. Each window is then Fourier transformed and the mean in frequency space is calculated. Figure \ref{fig:signal_response_example} shows an example of a chirp signal, the recorded equivalent and the impulse response in frequency space.

When looking at different phase plots of the RIR at the same position, it can be seen that the phase changes randomly. Because with the platform that is used to play the sound back, it is not possible to guarantee that the recording and sound playing starting at the same time. So there is always a random shift in time which depends on the CPU. If later needed, it may possible to compensate this shift by using cross-correlation to find the starting point of the chirp signal. But the error has to be significantly lower than 250 \si\micro s when using frequencies up to 4000 Hz which could be difficult because of the noisy recordings.

\begin{figure}[h!]
	\centering
	\captionsetup{justification=centering,margin=1cm}	
	\subfloat{{\includegraphics[width=0.5\textwidth]{images/signal_example.png} }}
	\subfloat{{\includegraphics[width=0.5\textwidth]{images/response_example.png} }}
	\caption{
		On the left in blue, the chirp signal with a frequency starting at 200 Hz and going up to 4000 Hz linearly. The chirp signal is 0.5 seconds long and was multiplied with a Tukey window to prevent clicking sounds. In orange the recorded signal and on the right the RIR.
	}
	\label{fig:signal_response_example}
\end{figure}

But it seems that the phase isn't necessary for this use case. When looking at the peaks of the magnitude, RIR at the same position will look similar as illustrated in figure \ref{fig:response-same-room-same-pos}. But that's expectable when looking at the work of Dokmanic \cite{dokmanic_roomrecslam_2016}. Also expectable is that the peaks in magnitude are at a different position when changing the room which can be seen in figure \ref{fig:response-different-room}. More interesting is the RIR of the same room but different positions. It turns out that the RIR is recognizably different in the same room but in different positions which can be seen in figure \ref{fig:response-same-room-different-pos}. Also interesting, peaks in the same room but different positions seem to be at similar positions when moving around (like in figure \ref{fig:dokmanic_roomrecslam}). But when changing the room the peak characteristic will change. That probably leads to a continuous function which is usable for SLAM purposes.

\begin{figure}[h!]
	\centering
	\captionsetup{justification=centering,margin=1cm}
	\includegraphics[width=0.49\textwidth]{images/response-same-room-same-pos.png}
	\caption{
		Two RIR of the position but different times. Notice, most of the peaks are at the same position.
	}
	\label{fig:response-same-room-same-pos}
\end{figure}

\begin{figure}[h!]
	\centering
	\captionsetup{justification=centering,margin=1cm}
	\includegraphics[width=0.49\textwidth]{images/response-different-room.png}
	\caption{
		Two RIR of different rooms. Notice, the peak characteristic has changed.
	}
	\label{fig:response-different-room}
\end{figure}

\begin{figure}[h!]
\centering
\captionsetup{justification=centering,margin=1cm}
\includegraphics[width=0.49\textwidth]{images/response-same-room-different-pos.png}
\caption{
	Two RIR of the same room but different positions. Notice, the change in peaks is recognizable but some peaks are only shifted slightly.
}
\label{fig:response-same-room-different-pos}
\end{figure}
